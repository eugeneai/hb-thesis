\documentclass[a4paper,14pt,final]{extreport}
\usepackage[microtyping,fancytop,handbook,cambriamath,cpcaption,firamono]{subook}
\usepackage{hyperref}
\usepackage[final]{minted}

\usemintedstyle{tango}
%\usemintedstyle{bw}
\setminted{breaklines=true,fontsize=\small,funcnamehighlighting=true}

\newcommand{\aaa}[1]{{/\bfseries #1\ldots/}}

\begin{document}
\thispagestyle{empty}
\begin{center}
  \small Министерство науки и образования\\
  Российской Фежерации \\
Институт математики, экономики и информатики\\
ФБУН <<Иркутский государственный университет>>
\end{center}
\vfill
\begin{center}
  Черкашин~Е.~А.\\[2em]
  {
  \Large\bfseries Как спроектировать текст\\
выпускной квалификационной работы\\
}
\vspace{1em}
\emph{учебное пособие}
\end{center}
\vspace{8em}
\vfill
\begin{center}
  Иркутск---2017
\end{center}
\mbox{}
\newpage

\tableofcontents

\chapter*{Введение}
\label{cha:inro}

Что-о о логической структуре ВКР, а также о логических связях в структуре информации.

Структура ВКР, в целом, соответствуе структуре жизненного цикла программного обеспечения (ЖЦ ПО) \cite{lifecycle}.

\chapter{Логическая структура ВКР}
\label{cha:logstru}

\chapter{Оформление текста}
\label{cha:layout}

Структура ВКР состоит из основных и дополнительных разделов.  К основным разделам относятся:
\begin{enumerate}
\item Титульный лист, на котором представлена информация формального характера:
  \begin{itemize}
  \item вуз, в котором выполнена работа,
  \item \aaa{кафедра},
  \item название ВКР,
  \item автор и научный руководитель;
  \end{itemize}
\item Введение, где изл....;
\item Глава, посвященная изложению теоретических аспектов работы;
\item Глава, где рассмотрены аспеты реализации программного продукта;
\item Заключение, в котором кратко перечисляются полученные результаты, \aaa{самокритика} и \aaa{направления дальнейшего совершенствования}.
\item Список используемых источников (список литературы).
\end{enumerate}

\section{Текстовые объекты}
\label{sec:objs}

\subsection{Формулы}
\label{sec:formulae}

Пример правильно оформленной формулы:
\begin{equation}
  \label{eq:1}
  \vec{F} = -G\frac{Mm}{|r^3|}\vec{r},
\end{equation}
где $\vec{F}$ -- вектор силы тяжести, действующий на материальную точку массой $m$ со стороны тела (материальной точки) массой $M$, находящегося в начале координат.  Радиус"=вектор $\vec{r}$ направлен из начала координат в цетр другого тела.  Уравнение Ньютона (\ref{eq:1}) в скалярной форме выглядит в следующем виде:
\begin{equation*}
  F=G\frac{Mm}{r^2},\quad r=|\vec{r}|, F=|\vec{F}|.
\end{equation*}

\subsection{Рисунки}
\label{sec:figures}

Рисунки в тексте отображаются в разрыве текста, по центру с нумерацией и подписью.
\begin{figure}[bh]
  \centering
  \def\svgwidth{0.8\linewidth}
  \input{bodies.pdf_tex}

%%  instead of
%%   \includegraphics[width=<desired width>]{<filename>.pdf}

  \caption{Ньютоновская система двух тел}
  \label{fig:bodies}
\end{figure}


\subsection{Алгоритмы}
\label{sec:algs}

\subsection{Тексты программ}
\label{sec:sources}

\begin{minted}{java}
/* HelloWorld.java
 */

public class HelloWorld
{
	public static void main(String[] args) {
		System.out.println("Hello World!");
	}
}
\end{minted}

\begin{minted}{csharp}
// Hello3.cs
// arguments: A B C D
using System;

public class Hello3
{
   public static void Main(string[] args)
   {
      Console.WriteLine("Hello, World!");
      Console.WriteLine("You entered the following {0} command line arguments:",
         args.Length );
      for (int i=0; i < args.Length; i++)
      {
         Console.WriteLine("{0}", args[i]);
      }
   }
}
\end{minted}

\chapter{Формальные части}
\label{cha:formal}

\section{Список литературы}
\label{sec:refs}

\begin{thebibliography}{99}
\bibitem{lifecycle} Жизненный цикл программного обеспечения --- Википедия. [электронный ресурс] URL:~\url{https://ru.wikipedia.org/wiki/%D0%96%D0%B8%D0%B7%D0%BD%D0%B5%D0%BD%D0%BD%D1%8B%D0%B9_%D1%86%D0%B8%D0%BA%D0%BB_%D0%BF%D1%80%D0%BE%D0%B3%D1%80%D0%B0%D0%BC%D0%BC%D0%BD%D0%BE%D0%B3%D0%BE_%D0%BE%D0%B1%D0%B5%D1%81%D0%BF%D0%B5%D1%87%D0%B5%D0%BD%D0%B8%D1%8F} (дата обращения: 28.04.2016)
\end{thebibliography}



\end{document}

%%% Local Variables:
%%% mode: latex
%%% TeX-master: t
%%% End:
