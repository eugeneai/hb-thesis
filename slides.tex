\documentclass{beamer}

%\usepackage[utf8]{inputenc}
\usepackage{fontspec}

\usepackage[english,russian]{babel}
%\usepackage[final]{hyperref}
\usepackage{unicode-math}


\setmainfont[Ligatures=TeX,
ItalicFont={Liberation Serif Italic},
BoldFont={Liberation Serif Bold},
BoldItalicFont={Liberation Serif Bold Italic}]{Liberation Serif}

% \setmonofont[
% Ligatures=TeX,
% Scale=MatchLowercase,Numbers=SlashedZero,
% Path=subook/Microsoft/,
% ItalicFont=couri.ttf,
% BoldFont=courbd.ttf,
% BoldItalicFont=courbi.ttf
% ]{cour.ttf}


% \setmainfont[Ligatures=TeX,
% Path=subook/Microsoft/,
% ItalicFont=timesi.ttf,
% BoldFont=timesbd.ttf,
% BoldItalicFont=timesbi.ttf]{times.ttf}
% \setmonofont[
% Ligatures=TeX,
% Scale=MatchLowercase,Numbers=SlashedZero,
% Path=subook/Microsoft/,
% ItalicFont=couri.ttf,
% BoldFont=courbd.ttf,
% BoldItalicFont=courbi.ttf
% ]{cour.ttf}
% \setsansfont[
% Ligatures=TeX,
% Scale=MatchLowercase,
% Path=subook/Microsoft/,
% ItalicFont=calibrii.ttf,
% BoldFont=calibrib.ttf,
% BoldItalicFont=calibriz.ttf
% ]{calibri.ttf}

% \setmathfont[
% Ligatures=TeX,
% Scale=MatchLowercase,
% Path=subook/Microsoft/
% ]{Cambria-Math.ttf}%

% \setmathfont[
% Ligatures=TeX,
% Path=\sbk@fontpath Euler/,
% Scale=MatchLowercase,
% math-style=upright,
% vargreek-shape=unicode
% ]{euler.otf}

%
% Choose how your presentation looks.
%
% For more themes, color themes and font themes, see:
% http://deic.uab.es/~iblanes/beamer_gallery/index_by_theme.html
%
\mode<presentation>
{
  \usetheme{Madrid}      % or try Darmstadt, Madrid, Warsaw, ...
  \usecolortheme{beaver} % or try albatross, beaver, crane, ...
  \usefonttheme{serif}  % or try serif, structurebold, ...
  \setbeamertemplate{navigation symbols}{}
  \setbeamertemplate{caption}[numbered]
}

%\usepackage[english]{babel}
%\usepackage[utf8x]{inputenc}
\usepackage{xcolor}
\usepackage{listings}
\lstset
{
    language=[LaTeX]TeX,
    breaklines=true,
    basicstyle=\tt\scriptsize,
    %commentstyle=\color{green}
    keywordstyle=\color{blue},
    %stringstyle=\color{black}
    identifierstyle=\color{magenta},
}

\title[JEEC 2015 Workshop]{\LaTeX{} для организации взаимодействия}
\author{Alexandre Bernardino}
\institute{ISR/IST}
\date{March 9, 2015}

\AtBeginSection[]
{
  \begin{frame}<beamer>
    \frametitle{Outline}
    \tableofcontents[currentsection,currentsubsection]
  \end{frame}
}

\begin{document}

\begin{frame}
  \titlepage
\end{frame}

% Uncomment these lines for an automatically generated outline.
\begin{frame}{Outline}
  \tableofcontents
\end{frame}

\section{Введение}

\begin{frame}{Motivation}
	\begin{itemize}
  		\item Большинство инженеров ленивы \ldots{}, и это, часто, хорошо
  		\begin{itemize}
			\item (\textit{lazy} $=$ \textit{to do things in the most efficient way})
		\end{itemize}
  		\pause
  		\item Engineers are terrible story tellers ... they prefer content to form
  		\pause
  		\item Readers are lazy ... need self contained and easy to read material
  		\pause
  		\item \LaTeX{} can help
  	\end{itemize}
\end{frame}

\begin{frame}{Why \LaTeX{} ?}
	\begin{itemize}
  		\item If everyone is lazy, why not use \textit{Word} / \textit{PowerPoint} ?
  		\pause
  		\item In \textit{Word} / \textit{PowerPoint} it is easy to make bad things.
  		\pause
  		\item In \LaTeX{} it is hard to do bad things.
  		\pause
  		\item \LaTeX{} automates structure and format so the author can focus on content.
  		\pause
  		\item \LaTeX{} keeps text, sections, figures, etc. globally well spaced using cool optimization algorithms!
  		\pause
  		\item \LaTeX{} is better to keep uniform the material contributed by different authors.
	\end{itemize}
\end{frame}

\section{Conclusion}

    \begin{frame}{Conclusion}
		\begin{columns}
			\begin{column}{5cm}
				\begin{figure}
   					%\includegraphics[height=4cm]{384px-KnuthAtOpenContentAlliance.jpg}
				\end{figure}
			\end{column}
			\begin{column}{5cm}
				\begin{flushright}
					\textit{The ideal situation occurs when
the things that we regard as beautiful
are also regarded by other
people as useful.}
					\vskip 0.5cm
				-- Donald Knuth
				\end{flushright}
			\end{column}
		\end{columns}
	\end{frame}

\end{document}

%%% Local Variables:
%%% mode: latex
%%% TeX-master: t
%%% End:
