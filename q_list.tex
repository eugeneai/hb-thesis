\documentclass[final,a4paper,12pt]{report}
\usepackage[utf8]{inputenc}
\usepackage[english,russian]{babel}
\begin{document}
\begin{center}
  \Large\bfseries Вопросы к экзамену по курсу <<Базы данных>>
\end{center}
\begin{enumerate}
\item Системный подход в моделировании программного обеспечения.
  \begin{enumerate}
  \item В чем состоит проблема сложности модели и пути ее решения.
  \item Понятия функционального блока, интерфейса и протокола.
  \item Принципы применения системного подхода в моделировнии
    программного обеспечения.
  \item Понятие жизненного цикла программного обеспечения. Этапы
    жизненного цикла.  Реализация системного подхода в проектировании
    программного обеспечения.
  \item Понятие парадигмы программирования. Охарактеризовать
    процедурную, функциональную и декларативную парадигмы
    программирования. Выделить основные их различия.  Выделить
    особенности параллельной, объектно"=ориентированной и компонентной
    парадигм программирования.
  \item Особенности сервис"=ориентированной парадигмы проектирования.
  \item Дать характеристику парадигме порождающего программирования.
  \item Показать положительные свойства парадигмы программирования,
    ориентированной на разработку языка описания предметной области.
  \item Выделить основные различия подходов к проектированию
    программного обеспечения: снизу"=верх, сверху"=вниз.
  \item Особенности экстремальных методов проектирования.  Перечислить
    методики моделирования информационных систем.
  \item Формальное преобразование моделей программного обеспечения как
    автоматизация различных этапов жизненного цикла программного
    обеспечения.
  \end{enumerate}
\item NoSQL"=Базы данных.
  \begin{enumerate}
  \item Области применения NoSql"=баз данных.
  \item Общее понятие NoSQL"=базы данных. Типы моделей данных. Способы
    организации информации в различных моделях данных. Достоинства и
    недостатки различных моделей данных.
  \item Выборка данных в NoSQL. Ограничения на значения
    атрибутов. Ограничение количества выбираемых строк. Сортировка.
  \item Понятие транзакции. Обработка транзакций. Состояние данных до
    и после команд обработки транзакций. Изоляция. Понятия ACID
    (Atomicity, Consistency, Isolation, Durability).
  \item Технология Memcache.
  \item Технология BerkeleyDB.
  \item Технология MongoDB.
  \item Технология SQLite.
  \item Методики полнотекстового тестирования (SphinxSearch,
    технологии PostgreSQL, MongoDB).
  \end{enumerate}
\item Язык программирования Python.
  \begin{enumerate}
  \item Особенности языка как языка программирования общего
    назначения.
  \item Синтаксис и семантика определений функций, классов, методов.
  \item Синтаксические структуры задания модулей.
  \end{enumerate}
\item Программные технологии компонентного проектирования.
  \begin{enumerate}
  \item Понятие компоненты как функционального блока сложной
    программной системы.
  \item Показать суть формального описания интерфейса
    компоненты. Схемы данных.
  \item Терминология компонентного проектирования: обслуживание
    (оснащение) интерфейса, реализация интерфейса; конфигурация. В чем
    отличие “обслуживания” от “реализации” интерфейса.
  \item Сервисы регистрации и загрузки компонент.
  \item Суть применения шаблонов проектирования при разработке
    компонентных приложений.
  \item Шаблонам “адаптер” и “фасад”.  Применение
    адаптер"=ориентированного подхода компонентного проектирования в
    разработке интерфейсов пользователя.
  \item Мультиадаптеры, подписчики и обработчики.
  \item Шаблон проектирования пользовательского интерфейса
    “Model"=View"=Controller”.
  \item Адаптер"=ориентированный подход в решении задач распределения
    компонент и защиты данных. Шаблон “заместитель” (Proxy).
  \item Особенности шаблона ”Фабрика классов”.
  \item Декларативные спецификации конфигурации компонентного
    приложения.
  \item Компонентные архитектуры ZCA, COM+/OLE2, XPCOM.
  \item Методики разработки распределенных программных систем при
    помощи сетевых протоколов.  Охарактеризовать свойства протоколов
    прикладного уровня: XML"=RPC и RESTful.
  \end{enumerate}
\item Логическое программирование.
  \begin{enumerate}
  \item Программные структуры ПРОЛОГ, факты, правила,
    запросы. Формализация структур в виде фраз Хорна.  Декларативная
    семантика пропозициональных вариантов фраз Форна.
  \item Процедурная семантика правил ПРОЛОГ.
  \item Перечислить элементарные типы данных.  Переменные и
    унификация.  Тема 3.  Определение концептуальной модели предметной
    области.
  \item Понятия: Данные и знания. Знания алгоритмические,
    эвристические. Стратегии.
  \item Сложные структуры данных ПРОЛОГ. Задание функторов.
  \item Рекурсивное представление списков. Базовые операции со
    списками.
  \item Изложить суть алгоритма Британского музея.
  \end{enumerate}
\item Искусственный интеллект.

  \begin{enumerate}
  \item Задачи ИИ, Виды обработки информации.
  \item Классификация задач искусственного интеллекта, их свойства.
  \item Представление знаний, формализмы представления знаний.
  \item Понятие “планирование действий”, допустимое состояние,
    допустимые переходы из состояние в состояние, цели, и т.~п.  Граф
    пространства состояний (ГПС).  Стратегии поиска решения без учета
    дополнительной информации.
  \item Стратегии поиска решения в ГПС с учетом дополнительной
    информации.  Понятия штрафов и стоимости решения,
    эвристик. Эвристический поиск. Алгоритм А*.
  \item Игры. Представление позиционных игр с полной информацией.
    Оценочные функции в игровых задачах.  Алгоритм
    MiniMax. Альфа"=бета отсечение.  Обход дерева MiniMax в
    глубину. Понятие горизонта. Сужение области поиска с помощью
    Альфа"=Бета отсечения.
  \item Экспертные системы. Структура экспертной
    системы. Классификация экспертных систем.  Принципы построения
    машин вывода экспертных систем.
  \item Программирование в терминах образцов.  Представление знаний в
    экспертных системах. Продукции.  Система CLIPS.
  \item Полнота базы знаний ЭС. Обработка неопределенности в
    экспертных системах.
  \item Эволюционные вычисления. Генетические алгоритмы.  Алгоритм
    муравья, алгоритм роя. Другие дискретные оптимизационные
    алгоритмы.  Градиентный спуск. Алгоритмы последовательного
    улучшения.
  \end{enumerate}
\end{enumerate}
\end{document}
